\mode*

% Since this a solution template for a generic talk, very little can
% be said about how it should be structured. However, the talk length
% of between 15min and 45min and the theme suggest that you stick to
% the following rules:  

% - Exactly two or three sections (other than the summary).
% - At *most* three subsections per section.
% - Talk about 30s to 2min per frame. So there should be between about
%   15 and 30 frames, all told.


\section{Some principles}

\subsection{Enkel autentisering}

\begin{frame}
  \begin{example}[Bättre fjärrlås]
    Låt \(A, B\) vara principals, \(n\) nonce, \(k_A\) en nyckel unik för 
    \(A\).
    \begin{align*}
      A\to B\colon A, \encrypt{A, n}{k_A}
    \end{align*}
  \end{example}
  \begin{block}{Egenskaper}
    \begin{itemize}
      \item Nonce \(n\) för färskhet.
      \item Krypteringen för identifiering.
    \end{itemize}
  \end{block}
\end{frame}

\begin{frame}{Kolla nonces}
  Kolla nonces långt tillbaka i tiden.
  \begin{itemize}
    \item Jämför med senaste nonce.
    \item Spela in två och spela upp dem varannan gång.
    \item Förbetalda elmätare, köp två laddningar och använd dem om vartannat.
  \end{itemize}
\end{frame}

\begin{frame}{Betjäntattacken}
  \begin{itemize}
    \item Hur genereras nonces?
    \item En person som har tillfällig åtkomst att generera tokens.
    \item Generera ett antal, använd dem senare.
    \item Exempelvis engångskoder för att logga in hos internetbanken.
    \item Attacken fungerar om nonces är (pseudo-) slumptal.
  \end{itemize}
\end{frame}

\begin{frame}{Kontra betjäntattacken}
  \begin{block}{Förbättring}
    \begin{itemize}
      \item Använd en räknare \(c\) som successivt ökas på.
      \item \(A\to B\colon A, \encrypt{A, c+1}{k_A}\), \(c = c+1\).
      \item Inget \(c^\prime \leq c\) accepteras.
    \end{itemize}
  \end{block}
  \begin{block}{Problem}
    \begin{itemize}
      \item Får inte ha jämförelsen \(c^\prime = c\), ger 
        synkroniseringsproblem.
      \item \(c\notin \Z_+\) utan \(c\in \Z_{2^x}\), för något \(x\in \N\): vid 
        något tillfälle blir då \(c+1 < c \pmod{2^x}\).
    \end{itemize}
  \end{block}
\end{frame}

\subsection{Challenge--response}

\begin{frame}
  \begin{block}{Grundläggande princip}
    Två principals \(A, B\) med gemensam nyckel \(k\) och nonce \(n\).
    \begin{align*}
      A\to B &\colon n \\
      B\to A &\colon \encrypt{B, n}{k}
    \end{align*}
  \end{block}
  \begin{block}{Problem}
    \begin{itemize}
      \item Dåliga (pseudo-) slumptalsgeneratorer, ger förutsägbara \(n\).
    \end{itemize}
  \end{block}
\end{frame}

\subsection{Miljöbyte}

\begin{frame}
  \begin{itemize}
    \item Betalkortsystemet designades för en pålitlig miljö.
    \item Kraftigt reglerad miljö inbyggd i bankens fasad.
    \item Tillämpas i den mindre pålitliga miljön i samtliga affärer.
    \item Skimming.
  \end{itemize}
\end{frame}

\begin{frame}{Personen i mitten}
  \begin{itemize}
    \item \enquote{Det är enkelt att spela oavgjort mot en schackstormästare 
        i postschack: spela bara mot två stormästare samtidigt, en som vit och 
        en som svart, och skicka deras brev mellan varandra.} (John Convey)
    \item Problem med pålitliga användargränssnitt: hur vet du att inte 
      kortterminalen ljuger?
  \end{itemize}
\end{frame}

\begin{frame}{Olika former av bankdosor}
  \begin{block}{Swedbank}
    \begin{itemize}
      \item Individuell dosa, förkonfigurerad av banken.
      \item Kan generera engångskod.
      \item Kan hantera challenge--response.
    \end{itemize}
  \end{block}

  \pause{}

  \begin{block}{Nordea}
    \begin{itemize}
      \item Oberoende smartkortläsare, använder individuellt betalkort.
      \item Kan generera engångskod.
      \item Kan hantera challenge--response.
    \end{itemize}
  \end{block}
\end{frame}

\begin{frame}{Problem som kan uppstå}
  \begin{block}{Problem}
    \begin{itemize}
      \item Om bankkort och dosa förvaras tillsammans kan PIN-koden utläsas 
        från de slitna knapparna på bankdosan.
      \item Om kortet används i en dålig terminal har angriparna allt som 
        behövs för att logga in till ditt bankkonto.
    \end{itemize}
  \end{block}

  \pause{}

  \begin{block}{Förbättringar}
    \begin{itemize}
      \item Använd inte samma säkerhetsmekanism i flera sammanhang.
      \item Ha separata oberoende mekanismer.
      \item Ha ett pålitligt användargränssnitt.
    \end{itemize}
  \end{block}
\end{frame}

% XXX add slides about BankID
%\subsection{BankID}
%\begin{frame}
%\end{frame}
%\begin{frame}{Att lämna in deklarationen}
%\end{frame}


%%%%%%%%%%%%%%%%%%%%%%

\begin{frame}
	\small
  \printbibliography
\end{frame}

