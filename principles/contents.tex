\mode*

% Since this a solution template for a generic talk, very little can
% be said about how it should be structured. However, the talk length
% of between 15min and 45min and the theme suggest that you stick to
% the following rules:  

% - Exactly two or three sections (other than the summary).
% - At *most* three subsections per section.
% - Talk about 30s to 2min per frame. So there should be between about
%   15 and 30 frames, all told.


\section{Some principles}

\subsection{Nonces}

\begin{frame}
  \begin{example}[Remote car locks]
    Let \(K\) be the key, \(C\) the car and \(k_{K, C}\) a shared key and \(n\) 
    is a nonce.
    \begin{align*}
      K\to C\colon K, \encrypt{K, n}{k_{K, C}}
    \end{align*}
  \end{example}

  \pause

  \begin{block}{Properties}
    \begin{itemize}
      \item Nonce \(n\) for freshness.
      \item Encryption for authentication.
    \end{itemize}
  \end{block}
\end{frame}

\begin{frame}
  \begin{question}
    \begin{itemize}
      \item How far back should we check nonces?
    \end{itemize}
  \end{question}

  \begin{example}
    \begin{itemize}
      \item Compare with the last nonce.
      \item Record two nonces and replay them every other time.
      \item Happened to prepaid electricity meters in South Africa.
    \end{itemize}
  \end{example}
\end{frame}

\begin{frame}
  \begin{question}
    \begin{itemize}
      \item How are the nonces generated?
    \end{itemize}
  \end{question}

  \begin{example}[Valet attack]
    \begin{itemize}
      \item Someone has temporary access.
      \item Generate \(m\) one-time tokens and use later.
      \item Works if nonces are (pseudo-)random.
    \end{itemize}
  \end{example}
\end{frame}

\begin{frame}
  \begin{example}
    \begin{itemize}
      \item Use a counter \(c\) which increases for every use.
      \item \(K\to C\colon K, \encrypt{K, c+1}{k_{K, C}}\), \(c\gets c+1\).
      \item No \(c' \leq c\) is accepted.
    \end{itemize}
  \end{example}

  \begin{remark}
    \begin{itemize}
      \item Checking \(c' = c\) yields synchronization errors.
      \item \(c\notin \Z_+\) but \(c\in \Z_{2^x}\), for some \(x\in \N\): \ie 
        \(c+1 < c \pmod{2^x}\) at some point.
    \end{itemize}
  \end{remark}
\end{frame}

\subsection{Challenge--response}

\begin{frame}
  \begin{definition}[Challenge--response]
    Two principals \(A, B\) with shared key~\(k\) and nonce \(n\).
    \begin{align*}
      A\to B &\colon n \\
      B\to A &\colon f_k(n)
    \end{align*}
    For some keyed function~\(f_k\).
  \end{definition}
\end{frame}

\begin{frame}
  \begin{example}[Man-in-the-middle]
    \begin{itemize}
      \item \enquote{It's easy to tie against a chess grand master in postal 
          chess: play against two and just relay their moves.}
    \end{itemize}
  \end{example}

  \pause

  \begin{example}
    \begin{itemize}
      \item The payment terminal in the supermarket is a man-in-the-middle 
        between you and your card.
    \end{itemize}
  \end{example}
\end{frame}

\begin{frame}
  \begin{example}
    \begin{itemize}
      \item All routers from you to your email provider are men-in-the-middle.
      \item Your phone operator is a man-in-the-middle between you and whoever 
        you call.
      \item That person's phone operator is also a man-in-the-middle.
      \item \dots
    \end{itemize}
  \end{example}

  \pause

  \begin{remark}
    \begin{itemize}
      \item Must protect against this.
      \item Can use public keys and digitally signed certificates.
      \item Where is the root of trust?
    \end{itemize}
  \end{remark}
\end{frame}

\subsection{Change of environment}

\begin{frame}
  \begin{example}[Bank card system for ATMs]
    \begin{itemize}
      \item Bank card system for ATMs.
      \item Secure environment in the facade of a bank.
      \item Move the system to stores for payment --- skimming!
    \end{itemize}
  \end{example}
\end{frame}

\begin{frame}
  \begin{example}[Swedbank banking hardware tokens]
    \begin{itemize}
      \item Hardware tokens are individualized, configured by bank.
      \item Generates one-time codes.
      \item Does challenge--response.
    \end{itemize}
  \end{example}

  \pause

  \begin{example}[Nordea banking hardware tokens]
    \begin{itemize}
      \item Non-individualized smart-card reader, individualized by payment 
        card.
      \item Generates one-time codes.
      \item Does challenge--response.
    \end{itemize}
  \end{example}
\end{frame}

\begin{frame}
  \begin{remark}
    \begin{itemize}
      \item If card reader and card stored together, maybe PIN can be read from 
        worn buttons.
    \end{itemize}
  \end{remark}
\end{frame}

\begin{frame}
  \begin{remark}
    \begin{itemize}
      \item What's the difference between a terminal in a supermarket and the 
        user's hardware token? Nothing.
      \item You enter PIN, they do the rest to log into your bank online.
      \item It's a man-in-the-middle, cannot tell the difference.
    \end{itemize}
  \end{remark}

  \pause

  \begin{remark}
    \begin{itemize}
      \item Hard to get away with this attack.
      \item Leave the country directly afterwards?
    \end{itemize}
  \end{remark}
\end{frame}

\begin{frame}
  \begin{block}{No-reuse principle}
    \begin{itemize}
      \item Don't reuse the same security mechanism.
      \item Have separate mechanisms.
      \item Have a trustworthy user interface.
    \end{itemize}
  \end{block}
\end{frame}

\begin{frame}
  \begin{example}[BankID]
    \begin{itemize}
      \item Trusted interface (phone, own computer), tells you what it's doing.
      \item Separate mechanisms for logging in and signing.
    \end{itemize}
  \end{example}

  \pause

  \begin{remark}
    \begin{itemize}
      \item Must be used correctly by the developer!
    \end{itemize}
  \end{remark}
\end{frame}



%%%%%%%%%%%%%%%%%%%%%%

\begin{frame}
	\small
  \printbibliography
\end{frame}

