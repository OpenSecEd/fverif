As soon as two principals need to interact, there is need for a protocol --- be 
it inside or between systems, even one principal communicating with itself in 
different points in time (which is the case when storing something for use at a 
later time).
These protocols need different properties.
We will explore how to design secure protocols and introduce some tools for 
verifying security properties of protocols.

More concretely, after this session you should be able to
\begin{itemize}
  \item \emph{understand} the different approaches and their limits to verify 
    the security of protocols.
\end{itemize}

Anderson gives an overview of this area in 
\citetitle{Anderson2008sea}~\cite{Anderson2008sea}, Chapter 
3 \enquote{Protocols}.
Gollmann has a more technically oriented treatment of a part of this topic in 
Chapter 15 of \citetitle{Gollmann2011cs}~\cite{Gollmann2011cs}.
To complement these texts we will also touch upon some of the material in 
\citetitle{ProVerif}~\cite{ProVerif}.
