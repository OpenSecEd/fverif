\mode*

% Since this a solution template for a generic talk, very little can
% be said about how it should be structured. However, the talk length
% of between 15min and 45min and the theme suggest that you stick to
% the following rules:  

% - Exactly two or three sections (other than the summary).
% - At *most* three subsections per section.
% - Talk about 30s to 2min per frame. So there should be between about
%   15 and 30 frames, all told.


\section{Secure protocols}

\subsection{What's a protocol?}

\begin{frame}
  \begin{definition}[Protocol]
    \begin{itemize}
      \item A system consists of some set of entities.
      \item A protocol is a set of rules governing their communication.
    \end{itemize}
  \end{definition}

  \pause

  \begin{example}[Exam]
    \begin{enumerate}
      \item The invigilator opens the room and gives each student a number.
      \item The students take their numbered seats.
      \item The invigilator checks each student's ID and number.
      \item The invigilator checks again when student hands in exam.
    \end{enumerate}
  \end{example}
\end{frame}

\begin{frame}
  \begin{remark}
    \begin{itemize}
      \item Protocols must be designed to resist attacks.
    \end{itemize}
  \end{remark}
\end{frame}

\begin{frame}
  \begin{example}[Ordering wine]
    \begin{enumerate}
      \item The Maître d'Hôtel brings the list of wines.
      \item The host chooses wine, the Maître d'Hôtel brings.
      \item The host tastes, then it's served.
    \end{enumerate}
  \end{example}

  \pause

  \begin{block}{Properties}
    \begin{description}
      \item[Confidentiality] The guests never learn the price.
      \item[Integrity] The Maître d'Hôtel cannot change wine.
      \item[Non-repudiation] The host cannot falsely complain about the wine 
        \emph{after dinner}.
    \end{description}
  \end{block}
\end{frame}

\subsection{Modelling and Assumptions}

\begin{frame}
  \begin{remark}
    \begin{itemize}
      \item Protocols are constructed from basic assumptions.
      \item We must analyse if the threats violates assumptions.
      \item Must analyse if the protocol actually mitigates them.
    \end{itemize}
  \end{remark}
\end{frame}

\begin{frame}
  \begin{example}[Exam]
    \begin{enumerate}
      \item The invigilator opens the room and gives each student a number.
      \item The students take their numbered seats.
      \item The invigilator checks each student's ID and number.
      \item The invigilator checks again when student hands in exam.
    \end{enumerate}
  \end{example}

  \pause

  \begin{block}{Properties}
    \begin{description}
      \item[Anonymity] The teacher cannot distinguish two students.
      \item[Authenticity] The student who handed in the exam gets the 
        grade.
    \end{description}
  \end{block}
\end{frame}

\begin{frame}
  \begin{block}{Properties}
    \begin{description}
      \item[Anonymity] The teacher cannot distinguish two students.
      \item[Authenticity] The student who wrote the exam gets the grade.
    \end{description}
  \end{block}

  \begin{exercise}
    \begin{itemize}
      \item Who is the adversary/-ies?
      \item What are the assumptions?
      \item Does the exam protocol achieve these properties?
    \end{itemize}
  \end{exercise}
\end{frame}

\begin{frame}
  \begin{remark}[Adversaries]
    \begin{itemize}
      \item The teacher wants to map student identity to exam.
      \item The student want to hand in someone else's answers.
    \end{itemize}
  \end{remark}

  \begin{question}
    \begin{itemize}
      \item What can they do?
    \end{itemize}
  \end{question}
\end{frame}

\begin{frame}
  \begin{remark}[Assumptions]
    \begin{itemize}
      \item The numbers are randomly assigned.
      \item The students cannot communicate during the exam.
    \end{itemize}
  \end{remark}
\end{frame}

\begin{frame}
  \begin{remark}[Deanonymize exams]
    \begin{itemize}
      \item The teacher had the same class in a previous course.
      \item The teacher learned everyone's handwriting.
      \item Handwriting is unique enough to deanonymize exams.
    \end{itemize}
  \end{remark}
\end{frame}

\subsection{More examples}

\begin{frame}
  \begin{example}[Non-anonymous exam]
    \begin{enumerate}
      \item The invigilator opens the room.
      \item The students take self-chosen seats.
      \item The invigilator checks each student's ID and notes placement.
      \item The student hands the exam to the invigilator.
    \end{enumerate}
  \end{example}

  \pause

  \begin{remark}
    \begin{itemize}
      \item No authenticity.
      \item Even with the same assumptions.
      \item Students can write each other's names on the exam.
      \item They don't need to communicate during the exam.
    \end{itemize}
  \end{remark}
\end{frame}

\begin{frame}
  \begin{example}[Authenticate withdrawals]
    \begin{itemize}
      \item Banks stored account number on the magnet strip.
      \item The PIN-code was sent to the central system for verification.
    \end{itemize}
  \end{example}

  \pause

  \begin{example}[Improved for offline ATMs]
    \begin{itemize}
      \item Banks stored account number on the magnet strip.
      \item Banks stored the PIN \emph{encrypted} on the magnet strip.
    \end{itemize}
  \end{example}
\end{frame}

\begin{frame}
  \begin{example}[Remote car locks, 1990s]
    \begin{itemize}
      \item The key broadcast the car's serial number.
      \item Wrongly assumed that only honest cars are listening.
      \item If in range, the car checked and unlocked.
    \end{itemize}
  \end{example}

  \pause

  \begin{remark}
    \begin{itemize}
      \item Replay attack: record, replay.
    \end{itemize}
  \end{remark}
\end{frame}

\begin{frame}
  \begin{example}[Remote car locks, 2019]
    \begin{itemize}
      \item Do secure crypto stuff (challenge--response).
    \end{itemize}
  \end{example}

  \pause

  \begin{remark}
    \begin{itemize}
      \item Relay attack: record in one end, transmit in other.
      \item Wrongly assumed that the key is close if it \enquote{talks} to the 
        car.
      \item Distance-bounding protocols are starting to get implemented.
    \end{itemize}
  \end{remark}
\end{frame}


\section{Some principles}

\subsection{Enkel autentisering}

\begin{frame}
  \begin{example}[Bättre fjärrlås]
    Låt \(A, B\) vara principals, \(n\) nonce, \(k_A\) en nyckel unik för 
    \(A\).
    \begin{align*}
      A\to B\colon A, \encrypt{A, n}{k_A}
    \end{align*}
  \end{example}
  \begin{block}{Egenskaper}
    \begin{itemize}
      \item Nonce \(n\) för färskhet.
      \item Krypteringen för identifiering.
    \end{itemize}
  \end{block}
\end{frame}

\begin{frame}{Nyckelhantering}
  \begin{itemize}
    \item Måste hantera nycklarna \(k_i\) för alla enheter \(i\).
    \item \emph{Nyckeldiversifiering}: huvudnyckel \(k_M\) och generera \(k_i 
      = \encrypt{i}{k_M}\).
    \item Måste tänka efter:
      \begin{itemize}
        \item 128-bitar nyckel krypterar 16-bitar ID, mindre lämpligt för 
          diversifiering.
        \item Svagt chiffer ger också dåligt resultat.
        \item \(k_i = i\xor k_M\)?
      \end{itemize}
  \end{itemize}
\end{frame}

\begin{frame}{Kolla nonces}
  Kolla nonces långt tillbaka i tiden.
  \begin{itemize}
    \item Jämför med senaste nonce.
    \item Spela in två och spela upp dem varannan gång.
    \item Förbetalda elmätare, köp två laddningar och använd dem om vartannat.
  \end{itemize}
\end{frame}

\begin{frame}{Betjäntattacken}
  \begin{itemize}
    \item Hur genereras nonces?
    \item En person som har tillfällig åtkomst att generera tokens.
    \item Generera ett antal, använd dem senare.
    \item Exempelvis engångskoder för att logga in hos internetbanken.
    \item Attacken fungerar om nonces är (pseudo-) slumptal.
  \end{itemize}
\end{frame}

\begin{frame}{Kontra betjäntattacken}
  \begin{block}{Förbättring}
    \begin{itemize}
      \item Använd en räknare \(c\) som successivt ökas på.
      \item \(A\to B\colon A, \encrypt{A, c+1}{k_A}\), \(c = c+1\).
      \item Inget \(c^\prime \leq c\) accepteras.
    \end{itemize}
  \end{block}
  \begin{block}{Problem}
    \begin{itemize}
      \item Får inte ha jämförelsen \(c^\prime = c\), ger 
        synkroniseringsproblem.
      \item \(c\notin \Z_+\) utan \(c\in \Z_{2^x}\), för något \(x\in \N\): vid 
        något tillfälle blir då \(c+1 < c \pmod{2^x}\).
    \end{itemize}
  \end{block}
\end{frame}

\begin{frame}{Andra tillämpningar}
  \begin{itemize}
    \item Tillbehörskontroll: skrivare ändrar inställning från \unit{1200}{dpi} 
      till \unit{300}{dpi} om icke-originalbläckpatroner används.
    \item \enquote{Använd alltid godkända originaldelar}.
    \item Inte hålla angripare ute, utan hålla användare inne.
    \item Läs kapitel 7 \emph{Economics} i~\cite{Anderson2008sea} för vidare 
      diskussion.
  \end{itemize}
\end{frame}

\subsection{Challenge--response}

\begin{frame}
  \begin{block}{Grundläggande princip}
    Två principals \(A, B\) med gemensam nyckel \(k\) och nonce \(n\).
    \begin{align*}
      A\to B &\colon n \\
      B\to A &\colon \encrypt{B, n}{k}
    \end{align*}
  \end{block}
  \begin{block}{Problem}
    \begin{itemize}
      \item Dåliga (pseudo-) slumptalsgeneratorer, ger förutsägbara \(n\).
    \end{itemize}
  \end{block}
\end{frame}

\begin{frame}{Tvåfaktorautentisering}
  \begin{itemize}
    \item Ha användarnamn och lösenord.
    \item Komplettera med extern kod; exempelvis genererad av koddosa, SMS till 
      mobiltelefonen.
    \item Finns många varianter, kombinera två:
      \begin{itemize}
        \item Något du vet (lösenord),
        \item något du har (koddosa, mobiltelefon),
        \item något du är (biometrik).
      \end{itemize}
  \end{itemize}
\end{frame}

\begin{frame}{Tvåfaktorautentisering}
  \begin{block}{Protokoll (tvåfaktorautentisering med koddosa)}
    Låt \(A, B, D\) vara principals, \(D\) är koddosa, \(k\) är nyckel delad 
    mellan \(B, D\) och \(p\) är \(A\):s PIN-kod.
    \begin{align*}
      A\to B &\colon A \\
      B\to A &\colon n \\
      A\to D &\colon n, p \\
      D\to A &\colon \encrypt{n}{k} \\
      A\to B &\colon \encrypt{n}{k}
    \end{align*}
  \end{block}
\end{frame}

\begin{frame}{Tvåkanalsautentisering}
  \begin{block}{Protokoll (tvåkanalsautentisering med mobiltelefon)}
    Låt \(A, B, M\) vara principals, \(M\) är mobiltelefon och \(p\) är \(A\):s 
    lösenord.
    \begin{align*}
      A\to B &\colon A, p \\
      B\to M &\colon n \\
      M\to A &\colon n \\
      A\to B &\colon n
    \end{align*}
  \end{block}
\end{frame}

\subsection{Miljöbyte}

\begin{frame}
  \begin{itemize}
    \item Betalkortsystemet designades för en pålitlig miljö.
    \item Kraftigt reglerad miljö inbyggd i bankens fasad.
    \item Tillämpas i den mindre pålitliga miljön i samtliga affärer.
    \item Skimming.
  \end{itemize}
\end{frame}

\begin{frame}{Personen i mitten}
  \begin{itemize}
    \item \enquote{Det är enkelt att spela oavgjort mot en schackstormästare 
        i postschack: spela bara mot två stormästare samtidigt, en som vit och 
        en som svart, och skicka deras brev mellan varandra.} (John Convey)
    \item Problem med pålitliga användargränssnitt: hur vet du att inte 
      kortterminalen ljuger?
  \end{itemize}
\end{frame}

\subsection{Internetbanken och betalkort}

\begin{frame}{Olika former av bankdosor}
  \begin{block}{Swedbank}
    \begin{itemize}
      \item Individuell dosa, förkonfigurerad av banken.
      \item Kan generera engångskod.
      \item Kan hantera challenge--response.
    \end{itemize}
  \end{block}

  \pause{}

  \begin{block}{Nordea}
    \begin{itemize}
      \item Oberoende smartkortläsare, använder individuellt betalkort.
      \item Kan generera engångskod.
      \item Kan hantera challenge--response.
    \end{itemize}
  \end{block}
\end{frame}

\begin{frame}{Problem som kan uppstå}
  \begin{block}{Problem}
    \begin{itemize}
      \item Om bankkort och dosa förvaras tillsammans kan PIN-koden utläsas 
        från de slitna knapparna på bankdosan.
      \item Om kortet används i en dålig terminal har angriparna allt som 
        behövs för att logga in till ditt bankkonto.
    \end{itemize}
  \end{block}

  \pause{}

  \begin{block}{Förbättringar}
    \begin{itemize}
      \item Använd inte samma säkerhetsmekanism i flera sammanhang.
      \item Ha separata oberoende mekanismer.
      \item Ha ett pålitligt användargränssnitt.
    \end{itemize}
  \end{block}
\end{frame}

% XXX add slides about BankID
%\subsection{BankID}
%\begin{frame}
%\end{frame}
%\begin{frame}{Att lämna in deklarationen}
%\end{frame}


%%%%%%%%%%%%%%%%%%%%%%

\begin{frame}
	\small
  \printbibliography
\end{frame}

